\justify
	The paper studies the method of using a ground slot for bandwidth improvement of compact ultra-wide band (UWB) antennas 	with microstrip line feed. Slots of different shapes such as triangular, rectangular, partially circular and hexagonal, placed on the ground plane under the feed line of the radiator are studied for impedance matching. \\

	
	\begin{flushleft}
		\textbf{Paper 1}
	\end{flushleft}

	\begin{center}
		   \begin{table}[h]

		   	\begin{tabular}{ |l|p{11cm}| }
		   		\hline
		   		Title of paper &  Bandwidth Improvements Using Ground Slots for Compact UWB Microstrip-fed Antennas \\
		   		\hline
		   		Authors & L. Liu, S. W. Cheung, and T. I. Yuk \\
		   		\hline
		   		Year of Publication & 2011 \\
		   		\hline
		   		Publishing details & International Conference on Education Technology and Computer (IEEE) \\ \hline
		   		Summary & A small microstip-fed monopole antenna, which consists of a rectangular patch and a truncated ground plane, is presented for ultra wideband application. The proposed antenna is designed to operate over 3.1 to 11 GHz for 11 10 dB. Good return loss and radiation pattern characteristics are obtained in the frequency band of interest.\\
		   		\hline
		   		Weblink & http://ieeexplore.ieee.org/stamp/stamp.jsp?tp=arnumber=5403292 \\
		   		\hline
		   	\end{tabular}

		   \end{table}
	\end{center}

	\begin{flushleft}
		\textbf{Paper 2}
	\end{flushleft}


	 \begin{center}
	 	\begin{table}[h]
	 		\centering
	 		\begin{tabular}{ |l|p{11cm}| }
	 			\hline
	 			Title of paper &  Planar UWB antenna with multi-slotted ground plane \\
	 			\hline
	 			Authors & Azim, R., M. T. Islam, N. Misran, S. W. Cheung, and Y. Yamada \\
	 			\hline
	 			Year of Publication & 2011 \\
	 			\hline
	 			Publishing details & The paper studies the method of using a ground slot for bandwidth improvement of compact ultra-wide band (UWB) antennas with microstrip line feed. Slots of different shapes such as triangular, rectangular, partially circular and hexagonal, placed on the ground plane under the feed line of the radiator are studied for impedance matching.\\
	 			\hline
	 			Weblink & http://ieeexplore.ieee.org/stamp/stamp.jsp?tp=arnumber=5403292 \\
	 			\hline
	 		\end{tabular}

	 	\end{table}
	 \end{center}

	\cleardoublepage
	
		\begin{flushleft}
			\textbf{Paper 3}
		\end{flushleft}
		

	  \begin{center}
	  	\begin{table}[h]
	  		\centering
	  		\begin{tabular}{ |l|p{11cm}| }
	  			\hline
	  			Title of paper &  Printed band-rejection UWB antenna with H-shaped slot \\
	  			\hline
	  			Authors & Bao, X. L. and M. J. Ammann \\
	  			\hline
	  			Year of Publication & 2012 \\
	  			\hline
	  			Publishing details & The ground element of the proposed antenna is taken in the form of defected ground structure. The antenna is fed by a microstrip feeding technique and printed on a dielectric Fr4 substrate of dimension (30mm X 32 mm) permittivity εr =4.4 and height h = 1.59 mm. The optimization on the microstrip has been done to accomplish an -10 dB return loss criterion. Design parameters like substrate variation, feed size and defected ground plane which affect the performance of the antenna in terms of its frequency domain and time domain characteristics are investigated.\\
	  			\hline
	  			Weblink & http://ieeexplore.ieee.org/stamp/stamp.jsp?tp=arnumber=5403292 \\
	  			\hline
	  		\end{tabular}

	  	\end{table}
	  \end{center}

	\begin{flushleft}
		\textbf{Paper 4}
	\end{flushleft}


		   	\begin{table}[h]
		   		\centering
		   		\begin{tabular}{ |l|p{11cm}| }
		   			\hline
		   			Title of paper &  Inverted triangle printed monopole antenna with halfdisk for UWB applications \\
		   			\hline
		   			Authors & Chayono, R., M. Haneishi, and Y. Kimura \\
		   			\hline
		   			Year of Publication & 2013 \\
		   			\hline
		   			Publishing details & International Conference on Education Technology and Computer (IEEE) \\ \hline
		   			Summary & A small microstip-fed monopole antenna, which consists of a rectangular patch and a truncated ground plane, is presented for ultra wideband application. The proposed antenna is designed to operate over 3.1 to 11 GHz for 11 10 dB. Good return loss and radiation pattern characteristics are obtained in the frequency band of interest.\\
		   			\hline
		   			Weblink & http://ieeexplore.ieee.org/stamp/stamp.jsp?tp=arnumber=5403292 \\
		   			\hline
		   		\end{tabular}

		   	\end{table}

  	  \cleardoublepage
  	  
		\begin{flushleft}
			\textbf{Paper 5}
		\end{flushleft}


	    	\begin{table}[h]
	    		\centering
	    		\begin{tabular}{ |l|p{11cm}| }
	    			\hline
	    			Title of paper & Small modified monopole antenna for UWB application \\
	    			\hline
	    			Authors & Ojaroudi, M., G. Kohneshahri, and J. Noory \\
	    			\hline
	    			Year of Publication & 2013 \\
	    			\hline
	    			Publishing details & International Conference on Education Technology and Computer (IEEE) \\ \hline
	    			Summary & A two-port rectangular microstrip patch antenna for dual frequency operation is investigated in this paper. Simple microstrip line feed has been used to feed the antenna. Quarter wavelength transformer is used for impedance matching. For the conventional dual feed dual frequency antenna the isolation between the two ports is obtained as 30 dB. An Improvement in isolation performance has been achieved by the introduction of defected microstrip structure which acts as band stop filters and thereby increases isolation between the two ports.\\
	    			\hline
	    			Weblink & http://ieeexplore.ieee.org/stamp/stamp.jsp?tp=arnumber=5403292 \\
	    			\hline
	    		\end{tabular}

	    	\end{table}
	    	
	    	\begin{flushleft}
	    		\textbf{Paper 6}
	    	\end{flushleft}
	    	

		  \begin{center}
		  	\begin{table}[H]
		  		\centering
		  		\begin{tabular}{ |l|p{11cm}| }
		  			\hline
		  			Title of paper &  Development of a practical ultra- wideband antenna with planar circuit integration possibilities \\
		  			\hline
		  			Authors & G. Brzezina , Q. Ye , L. Roy \\
		  			\hline
		  			Year of Publication & 2013 \\
		  			\hline
		  			Publishing details & International Conference on Education Technology and Computer (IEEE) \\ \hline
		  			Summary & The printed antenna is one of the best antenna structures, due to its low cost and compact design. In this paper a new approach to improve the radiation effectiveness and the performance of antennas by miniaturization of the size. Indeed, we have studied the performance of ultra wideband antenna which consists of a ring-shaped patch. This study was made for the whole frequency band of UWB ranging from 2.5GHz to 9.4GHz and the geometry of the antenna and the results were obtained using the simulation software. \\
		  			\hline
		  			Weblink & http://ieeexplore.ieee.org/stamp/stamp.jsp?tp=arnumber=5403292 \\
		  			\hline			 
		  		\end{tabular}		
		  		
		  	\end{table}
		  \end{center}
		  
		\cleardoublepage
			
			\begin{flushleft}
				\textbf{Paper 7}
			\end{flushleft}
			

		  \begin{center}
		  	\begin{table}[H]
		  		\centering
		  		\begin{tabular}{ |l|p{11cm}| }
		  			\hline
		  			Title of paper & Comparison between Straight and U shape of ultra-wideband microstrip antenna using log periodic technique  \\
		  			\hline
		  			Authors & M. K. A. Rahim, M. N. A. Karim, T. Masri, A. Asrokin \\
		  			\hline
		  			Year of Publication & 2013 \\
		  			\hline
		  			Publishing details & International Conference on Education Technology and Computer (IEEE) \\ \hline
		  			Summary & A band notched ultra-wideband (UWB) patch antenna is presented with its circuit modeling. The rectangular patch antenna is designed on dielectric substrate and fed with 50 Ω microstrip by optimizing the width of partial ground, the width of the feed line to operate in UWB. This antenna consists of a radiating element with a strip, and a partial ground plane and feeding line has been demonstrated. With the design, the return loss is lower than 10 dB in 3.1-10.6 GHz frequency range and show the band-notch characteristic in the UWB band to avoid interferences, which is caused by WLAN (5.15–5.825 GHz) and WiMax (5.25–5.85 GHz) systems. \\
		  			\hline
		  			Weblink & http://ieeexplore.ieee.org/stamp/stamp.jsp?tp=arnumber=5403292 \\
		  			\hline			 
		  		\end{tabular}		
		  		
		  	\end{table}
		  \end{center} 
				 
			\begin{flushleft}
				\textbf{Paper 8}
			\end{flushleft}
			

		  \begin{center}
		  	\begin{table}[H]
		  		\centering
		  		\begin{tabular}{ |l|p{11cm}| }
		  			\hline
		  			Title of paper &  Wide band high efficiency printed loop antenna design for wireless communication systems  \\
		  			\hline
		  			Authors & Denidni, T.A., L., Lim, Y., and Rao, Q \\
		  			\hline
		  			Year of Publication & 2014 \\
		  			\hline
		  			Publishing details & International Conference on Education Technology and Computer (IEEE) \\ \hline
		  			Summary & A fractal monopole antenna is proposed for the application in the UWB frequency range, which is designed by the combination of two fractal geometries.A fractal micro strip antenna is used for multiband application in this project provides a simple and efficient method for obtaining the compactness. A sierpinski carpet based fractal antenna is designed for multiband applications. It should be in compactness and less weight is the major point for designing an antenna. This antenna is providing better efficiency.\\
		  			\hline
		  			Weblink & http://ieeexplore.ieee.org/stamp/stamp.jsp?tp=arnumber=5403292 \\
		  			\hline			 
		  		\end{tabular}		
		  		
		  	\end{table}
		  \end{center}
		
		\cleardoublepage
		
			\begin{flushleft}
				\textbf{Paper 9}
			\end{flushleft}
			
	
		  \begin{center}
		  	\begin{table}[H]
		  		\centering
		  		\begin{tabular}{ |l|p{11cm}| }
		  			\hline
		  			Title of paper &  Design of reconfigurable slot antenna  \\
		  			\hline
		  			Authors & Peroulis, D., Sarabi, K., and Katehi, L.P.B \\
		  			\hline
		  			Year of Publication & 2014 \\
		  			\hline
		  			Publishing details & The characteristics of a small antenna using an H-shaped microstrip patch antenna are analyzed. Operating frequency of H-shaped microsrip antenna is 2 GHz. The theoretical results are compared with experimental result using cavity model. Comparison with other reported results justify the veracity of the proposed method. Significant reduction of antenna size can be realized when the H-shaped patch is used instead of the conventional rectangular microstrip patch antenna. This antenna is providing better efficiency.\\
		  			\hline
		  			Weblink & http://ieeexplore.ieee.org/stamp/stamp.jsp?tp=arnumber=5403292 \\
		  			\hline			 
		  		\end{tabular}		
		  		
		  	\end{table}
		  \end{center}
		  
			 	\begin{flushleft}
			 		\textbf{Paper 10}
			 	\end{flushleft}
			 	
		 
		 
		    \begin{center}
		    	\begin{table}[H]
		    		\centering
		    		\begin{tabular}{ |l|p{11cm}| }
		    			\hline
		    			Title of paper &  Design of band notched UWB patch antenna with circular slot  \\
		    			\hline
		    			Authors & Shilpa Jangid and Mithilesh Kumar \\
		    			\hline
		    			Year of Publication & 2011 \\
		    			\hline
		    			Publishing details & Different feeding techniques of microstrip patch antennas with different spiral defected ground structures are presented in this paper. The investigated structures illustrate some merits in designing multi-electromagnetic band-gap structures by adjusting the capacitance and changing the inductance through varying the width and length of spiral defected ground structure.\\
		    			\hline
		    			Weblink & http://ieeexplore.ieee.org/stamp/stamp.jsp?tp=arnumber=5403292 \\
		    			\hline			 
		    		\end{tabular}		
		    		
		    	\end{table}
		    \end{center}
		    
		    \cleardoublepage
		    
		    	\begin{flushleft}
		    		\textbf{Paper 11}
		    	\end{flushleft}
		    	

		      \begin{center}
		      	\begin{table}[H]
		      		\centering
		      		\begin{tabular}{ |l|p{11cm}| }
		      			\hline
		      			Title of paper &  Microstrip Antenna gain enhancement using left-handed metamaterial structure \\
		      			\hline
		      			Authors & H.A. Majid, M.K.A. Rahim and T. Marsi \\
		      			\hline
		      			Year of Publication & 2015 \\
		      			\hline
		      			Publishing details & International Conference on Education Technology and Computer (IEEE) \\ \hline
		      			Summary & This paper describes the effect of temperature variation on microstrip patch antenna for different substrate materials. Eight materials are chosen as substrate and the effect of temperature variation is studied on each substrate material. A technique of temperature compensation has also been developed with substrate height variation. It is also seen that the change in resonance frequency due to variation of temperature can be compensated by varying the height of the substrate. The proposed antenna is designed and simulated by using HFSS software.\\
		      			\hline
		      			Weblink & http://ieeexplore.ieee.org/stamp/stamp.jsp?tp=arnumber=5403292 \\
		      			\hline			 
		      		\end{tabular}		
		      		
		      	\end{table}
		      \end{center}
		      
		      	\begin{flushleft}
		      		\textbf{Paper 12}
		      	\end{flushleft}
		      	

		       \begin{center}
		       	\begin{table}[H]
		       		\centering
		       		\begin{tabular}{ |l|p{11cm}| }
		       			\hline
		       			Title of paper &  Single-feed dual-frequency rectangular microstrip antenna with square slot \\
		       			\hline
		       			Authors & Chen W. S. \\
		       			\hline
		       			Year of Publication & 2015 \\
		       			\hline
		       			Publishing details & International Conference on Education Technology and Computer (IEEE) \\ \hline
		       			Summary & This paper presents the result for antenna factor of microstrip patch antenna when used as electromagnetic interference (EMI) sensor. Antenna factor is an important parameter of a sensor used for EMI measurements. The microstrip antenna has found wide application as transmit and receive antenna in modern microwave systems. In this paper, a new application of microstrip antenna as EMI sensor is presented. The result for antenna factor versus frequency of a microstrip patch antenna is presented using commercial software CST Microwave Studio. Also the experimental results for a prototype antenna are presented and compared with the simulated result.\\
		       			\hline
		       			Weblink & http://ieeexplore.ieee.org/stamp/stamp.jsp?tp=arnumber=5403292 \\
		       			\hline			 
		       		\end{tabular}		
		       		
		       	\end{table}
		       \end{center}		 
		
		\cleardoublepage
		
			\begin{flushleft}
				\textbf{Comparision}
			\end{flushleft}
			

		\begin{center}
			\begin{table}[H]
				\begin{tabular}{ |p{8cm}|p{7cm}|}
					\hline
					PAPER & RESULT \\ \hline
					
					Bandwidth Improvements Using Ground Slots for Compact UWB Microstrip-fed Antennas & The hexagonal slot provides the largest impedance bandwidth of 3.1-16.3 GHz for S11 < ¡10 dB, with an average gain of about 2.8 dBi and an average efficiency of about 88\% \\ \hline
					
					Development of a practical ultra- wideband antenna with planar circuit integration possibilities & This antenna can also be operated at 2.478 GHz as it provides dual band operation. At 2.478 GHz the values of Return loss and bandwidth are -30.218dB and 33.1 MHz respectively \\ \hline
					
					Wide band high efficiency printed loop antenna design for wireless communication systems & A compact, 31mm x 21mm low profile planar ultra-wide band patch antenna was introduced. The antenna was ex-cited using a rectangular edge-feed microstrip feed line. The impedance bandwidth of the antenna is about 11 GHz (3.0-14GHz), which exceeds the FCC UBW requirement. \\ \hline
					
					Design of reconfigurable slot antenna & Besides exhibiting a 10-dB bandwidth of 172\% with 13.06:1 ratio bandwidth, a 14-dB bandwidth (low return loss) of 79\% is also demonstrated in the higher UWB operating bands for outdoor propagation.\\ \hline
					
					Design of band notched UWB patch antenna with circular slot  & The patch antenna ring ultrawide-bandwidth radiating between 2.5GHz and 9.4GHz in order to achieve the
					operation Bluetooth / ISM, 2.5/3.5 GHz and 5.2/5.7 GHz WiMAX WLAN. \\ \hline
					
					Microstrip Antenna gain enhancement using left-handed metamaterial structure & The resonant frequency of the antenna is 1.99 GHz.
					return loss with frequency of antenna is found to be -30.33 dB at resonant frequency 1.99 GHz. \\ \hline
					
					Single-feed dual-frequency rectangular microstrip antenna with square slot &  At 2GHz the verified and tested result on RadiationEfficiency=91.99\%, Directivity=5.4dBi,Directive gain=4.98dBi \\ \hline
					
					
					
				\end{tabular}
			\end{table}
		\end{center}


\cleardoublepage