% This is the objective, need and organiztion


\subsection{Objective}\label{subs:Objective}
 \justify
   A  rectangular microstrip antenna is a type of radio antenna with a low profile, which can be mounted on a flat surface. It consists of a flat rectangular sheet of metal, mounted over a larger sheet of metal called a ground plane.The two metal sheets together form a resonant piece of microstrip transmission line with a length of approximately one-half wavelength of the radio waves. The radiation mechanism arises from discontinuities at each truncated edge of the microstrip transmission line.These patch antennas are used as simple and for the widest and most demanding applications. Dual characteristics, circular polarizations, dual frequency operation, frequency agility, broad band width, feed line flexibility and beam scanning can be easily obtained from these patch antennas.

 \justify
  Rectangular Microstrip Patch Antenna with High Gain for 3.1 GHz - 10.6 GHz Applications development of low cost, minimal weight and low profile antennas that are capable of maintaining high performance over a wide spectrum of frequencies. This technological trend has focused much effort into the design of a microstrip patch antenna. The objective of this paper is to design, and fabricate an inset fed rectangular microstrip patch antenna.

\subsection{Need}\label{sub:Need}
 \justify
   Microstrip patch antennas are increasing in popularity for use in wireless applications due to their low-profile structure. Therefore they are extremely compatible for embedded antennas in handheld wireless devices such as cellular phones, pagers etc... The telemetry and Square Rectangular Dipole Circular Triangular Circular Ring Elliptical communication antennas on missiles need to be thin and conformal and are often Microstrip patch antennas. Another area where they have been used successfully is in Satellite communication.

 \justify
   Microstrip patch antennas have a very high antenna quality factor (Q). Q represents the losses associated with the antenna and a large Q leads to narrow bandwidth and low efficiency. Q can be reduced by increasing the thickness of the dielectric substrate.

 \justify
   As per requirement many new shapes can replace the conventional shapes .There are many shapes in the field of microstrip patch antenna .A design of slots on the patch and making defected structure in the ground plane for improving the bandwidth as well as achieving the multiband operation which is the part of this project is very good for future aspects. All works has been performed in the thesis with the HFSS simulation software.

\cleardoublepage


\subsection{Organisation}\label{sub:Organisation}
 \justify
  \textbf{Chapter 1}

      It contains the overall introduction to the rectangular microstrip patch antenna.In this chapter also concluded with the details of outline of the present report .\
 \justify
  \textbf{Chapter 2}

      It is dedicated to literature survey which gives an overview about microstrip patch antenna based on various international publish papers on IEEE, google scholar etc.
 \justify
  \textbf{Chapter 3}

      It is based on totally system modeling i.e. total description of technical parts related to topic.The basic parameters on which the selection and performance of an Antenna is characterize are bandwidth, Antenna polarization, Radiation , Beamwidth ,Pattern, Directivity, efficiency etc are described. All the popular feeding methods used in microstrip antenna with their significance are also discussed.
 \justify
  \textbf{Chapter 4}

      It contains Advantages and Applications of Microstrip Patch Antenna.
 \justify
  \textbf{Chapter 5}

      It contains the conclusion of the project report.
\cleardoublepage